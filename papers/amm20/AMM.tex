\documentclass{article}


\usepackage{arxiv}

\usepackage[utf8]{inputenc} % allow utf-8 input
\usepackage[T1]{fontenc}    % use 8-bit T1 fonts
\usepackage{hyperref}       % hyperlinks
\usepackage{url}            % simple URL typesetting
\usepackage{booktabs}       % professional-quality tables
\usepackage{amsfonts}       % blackboard math symbols
\usepackage{nicefrac}       % compact symbols for 1/2, etc.
\usepackage{microtype}      % microtypography
\usepackage{lipsum}
\usepackage{etoolbox}
\usepackage{amsmath}
\usepackage{subcaption}
\usepackage{graphicx}

\graphicspath{{images/}}
\captionsetup[figure]{labelfont={large},textfont={large}}
\captionsetup[subfigure]{labelfont={}, textfont={}}

\title{Yield Farming with Autopilot}


\author{
  Chris Slaughter
  LVL\\
  Los Angeles, CA 900036 \\
  \texttt{chris@lvl.co}
}

\begin{document}
\maketitle

\begin{abstract}

The role of a market maker is to provide liquidity on an exchange by quoting bid and ask prices for a small discount or premium to the market price. Automated market making is an active strategy that continuously makes the market in one or more assets. Automated market making has been extensively studied, and an optimal market making algorithm has been proposed by Avellaneda \& Stoikov (2008). In this paper, we review the state of the art in optimal market making, and then propose a simplified market making model that relies on only two parameters: spread and give. We also demonstrate the performance of the model in simulations.

\end{abstract}

\keywords{Exchange \and Liquidity \and Automated Market Making}

\section{Introduction}
\label{sec:intro}



\bibliographystyle{unsrt}  

\begin{thebibliography}{1}



\end{thebibliography}


\end{document}
